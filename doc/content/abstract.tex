\begin{abstract}
In dieser Arbeit wird untersucht, wie Punkt- und Liniendiagramme durch Interaktion verbessert werden können, sodass sich der Benutzer effektiver mit dem dargestellten Datensatz auseinandersetzen kann und die Analyse und das Verständnis der Daten erleichtert wird. 

Eine Webapplikation für die Darstellung von 2-, 3- und mehrdimensionalen Datensätzen wird entwickelt. Die Technologie einer Webapplikation wird vorgestellt (JavaScript, HTML, CSS, SVG), das Applikationsdesign (Benutzeroberfläche, Konfiguration, Filtering, Mapping, Rendering) und Probleme, die während der Entwicklung auftraten, werden dokumentiert. Diese Arbeit erläutert Konzepte der Datenvisualisierung (Visualisierungspipeline) und des Designs von Benutzeroberflächen (Information-Seeking Mantra).

Bei den entwickelten interaktiven Diagrammen wurden verschiedene Interaktionsmethoden implementiert, die geläufig sind (wie z.B. Zoom, Filter, Details auf Abruf) oder andererseits zuvor nicht verwendet wurden (Reduktion von drei- auf zweidimensionale Punktediagramme durch Projektion).
\end{abstract}