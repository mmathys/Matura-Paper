\begin{abstract}
In dieser Arbeit wurde untersucht, wie Punkt- und Liniendiagramme durch Interaktion verbessert werden können, sodass sich der Benutzer effizienter mit dem dargestellten Datensatz auseinandersetzen kann und die Analyse und das Verständnis der Daten erleichtert werden. 

Eine Webapplikation für die Darstellung von 2- und 3-dimensionalen Datensätzen wurde entwickelt. Die Technologie einer Webapplikation wurde vorgestellt (JavaScript, HTML, CSS, SVG), das Applikationsdesign (Benutzeroberfläche, Konfiguration, Filtering, Mapping, Rendering) und Probleme, die während der Entwicklung auftraten, wurden dokumentiert. In dieser Arbeit wurden Konzepte der Datenvisualisierung (Visualisierungspipeline) und des Designs von Benutzeroberflächen (Information-Seeking Mantra) erläutert.

Bei den entwickelten interaktiven Diagrammen wurden verschiedene Interaktionsmethoden implementiert, die geläufig sind (wie z.B. Zoom, Filter, Details auf Abruf) oder in der Praxis nur selten verwendet werden (Reduktion von drei- auf zweidimensionale Punktdiagramme durch Projektion).
\end{abstract}