\begin{abstract}
In dieser Arbeit wird untersucht, wie Punkte- und Liniendiagramme durch Interaktion verbessert werden können, sodass sich der Benutzer effektiver mit dem dargestellten Datensatz auseinandersetzen kann und die Analyse und das Verständnis der Daten erleichtert wird. 

Eine Webapplikation für die Darstellung von 2-, 3- und mehrdimensionalen Datensätzen wird entwickelt. Die Technologie einer Webapplikation wird vorgestellt (JavaScript, HTML, CSS, SVG), das Applikationsdesign (Benutzeroberfläche, Konfiguration, Filtering, Mapping, Rendering) und Probleme, die während der Entwicklung auftraten, werden dokumentiert. Konzepte der Datenvisualisierung (Visualisierungspipeline) und des Designs von Benutzeroberflächen (Information-Seeking Mantra) werden erläutert.

Bei den entwickelten interaktiven Diagramme wurden verschiedene Interaktionsmethoden implementiert, die einerseits geläufig sind (wie z.B. Zoom, Filter, Details auf Abruf) oder zuvor nicht verwendet wurden (Reduktion von drei- auf zweidimensionale Punktediagramme durch Projektion).
\end{abstract}