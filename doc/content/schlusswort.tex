\chapter{Schlusswort}

Durch Umsetzung von verschiedenen Interaktionsmethoden an selbstentwickelten Diagrammen konnte die Analyse, Verständnis an zwei von drei Diagrammtypen gegenüber der statischen Version verbessert werden.

Neben der Dokumentation des Entwicklungsprozesses und der verwendeten Technologien wurde für die interaktiven Diagramme eine optimale Datenverarbeitungsstragie entwickelt (siehe "`Merge"' und "`meta.json"'). Nebenbei wurde auch viel Wissen über JavaScript im Allgemeinen, JavaScript Buildsystemen (NPM, Gulp.js, Browserify), Open Source, Web Design (CSS, Minimalismus, Pop/Unpop) und über den Umgang mit Programmbibliotheken (D3, three.js, tween.js) erarbeitet.

Beim zweidimensionalen Punkt-/Liniendiagamm wurden Möglichkeiten für die Interaktion mit Benutzeroberflächen entsprechend implementiert: Zoom, Tooltip, Detailanzeige, Datensatzauswahl. Es wurde so auch erreicht, dass Benutzer sich mit einer grossen Datenmenge effizienter auseinandersetzen können. Zudem wurden Diagrammtechniken implementiert wie Achsen, Skalierungen, Interpolationen. Die Theorie hinter diese Möglichkeiten und Techniken sind ebenfalls in der Arbeit dokumentiert worden.

Eine unkonventionelle Weise der Darstellung eines Datensatzes mit zwei abhängigen Variablen, das dreidimensionale Diagramm, wurde erläutert und als Applikation umgesetzt. Eine Methode wurde dazu entwickelt, die die Benutzung des Diagramms produktiver machen sollte: Orthographische Projektionen, die das dreidimensionale Punktdiagramm auf zweidimensionale Punktdiagramme reduzieren können.

Ein N-dimensionales Punktdiagramm wurde umgesetzt, doch es wurde erkannt, dass sich dieses Diagramm nicht als interaktives Diagramm eignet.

