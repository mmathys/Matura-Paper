\chapter{Hauptteil}

% 1. voraussetzung für diagramm: daten.

%Um ein Diagramm zu erstellen, benötigt man einen \textit{Datensatz}, das die Informationen bereitstellt. In den Beispielen dieser Arbeit werden Daten in \textit{zwei-}, \textit{drei-} und \textit{n-dimensionaler Form} verwendet: Es sind je ein, zwei oder $n-1$ unabhängige Variablen vorhanden und eine abhängige Variable. 

%Die Daten werden in \textit{relationaler Struktur} gespeichert und vom Programm verarbeitet. Eine relationale Strukturierung ist die Speicherung von Merkmalen und ihren Ausprägungen in Tabellenform \cite{viz}.

%% 1.1. erwähnen: abhängige und unabhängige variable. verschieden, wenn verschiedene dimensionen.

% 1.2 relationale struktur

% 1.3 datenquellen: reale welt, theoretische welt, künstliche welt. reale welt hier.