\section{Verarbeitung der Daten}

Bevor das Diagramm im Browser dargestellt werden kann, müssen zuerst die Daten geladen verarbeitet werden. Die Beispiele dieser Arbeit nutzen grösstenteils Datensätze, die frei verfügbar sind.  %hier näher erklären TODO
Die Diagramme in Abbildung \ref{fig:scatterplot} benutzen einen Datensatz der Weltbank, der den $CO_2$-Verbrauch von Afghanistan beinhaltet.

Beim Prozess zur Veranschaulichung von Daten wird die \textit{Visualisierungspipeline} durchlaufen \cite{pipeline}. Diese Pipeline stellt drei wesentliche Schritte des Prozesses dar: Die Datenaufbereitung (\textit{Filtering}), die Erzeugung des Geometriemodells (\textit{Mapping}) und die Bildgenerierung (\textit{Rendering}).

% = rohdaten -> aufbereitete Daten -> geometriedaten -> Bilddaten

\subsection{Rohdaten}

\subsection{Filtering}

\subsection{Aufbereitete Daten}

\subsection{Mapping}

\subsection{Geometriedaten}

\subsection{Rendering}

\subsection{Bilddaten}