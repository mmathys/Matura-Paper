\section{Verarbeitung der Daten}

Bevor das Diagramm im Browser dargestellt werden kann, müssen zuerst die Daten geladen und verarbeitet werden. Die Beispiele dieser Arbeit nutzen ausschliesslich Datensätze, die frei verfügbar sind \cite{worldbank}.  %hier näher erklären TODO
Die Diagramme in Abbildung \ref{fig:scatterplot} benutzen einen Datensatz der Weltbank, der den $CO_2$-Verbrauch von Afghanistan beinhaltet.

%TODO beschreiben wieso ich diesen teil so ausführlich beschreibe. weil die datenverarbeitung überraschend schwierig umzusetzen war.darum will ich zeigen wieso.

Beim Prozess zur Veranschaulichung von Daten wird die \textit{Visualisierungspipeline} durchlaufen \cite[Kapitel 2.1]{viz}. Diese Pipeline stellt drei wesentliche Schritte des Prozesses dar: Die Datenaufbereitung (\textit{Filtering}), die Erzeugung des Geometriemodells (\textit{Mapping}) und die Bildgenerierung (\textit{Rendering}).

\subsection{Rohdaten}

\subsubsection*{Comma-separated values (CSV)}

Die Applikation verwendet als Datenformat \textit{Comma-separated values} (\textit{CSV}). Das CSV-Format wird verwendet, um Tabellendaten zwischen Programmen auszutauschen \cite{csv}. Tabellenzeilen werden in der CSV-Datei durch einen Umbruch dargestellt, Spaltenwerte durch Kommas. Optional steht in der ersten Zeile (Abbildung \ref{fig:csv} links, Zeile 1) der Datei die Beschriftung der Spalten.

In Abbildung \ref{fig:csv} wird die Funktion des CSV-Formats demonstriert.

\begin{figure}[!htbp]
	\centering
	\begin{minipage}{0.5\textwidth}
		\centering
		\begin{minted}[linenos]{text}
Date,Value
2010-12-31,4220.717
2009-12-31,4352.729
2008-12-31,5555.505
2007-12-31,5067.794
		\end{minted}
	\end{minipage}\hfill
	\begin{minipage}{0.5\textwidth}
		\centering
		\begin{tabular}{ | l | l |}
			\hline
			\textbf{Date} & \textbf{Value} \\ \hline
			2010-21-31 & 4220.717 \\ \hline
			2009-21-31 & 4352.729 \\ \hline
			2008-21-31 & 5555.505 \\ \hline
			2007-21-31 & 5067.794 \\ \hline
		\end{tabular}
	\end{minipage}
	\caption[Demonstration des CSV-Formats]{Demonstration des CSV-Formats. Links: Die CSV-Datei in Rohtext. Rechts: Darstellung der Informationen der CSV-Datei in einer Tabelle.}
	\label{fig:csv}
\end{figure}

Man verwendet das CSV-Format oft, weil es sehr einfach aufgebaut ist. Das Lesen von solchen Tabellenformaten ist ohne grossen Aufwand in Programmen umsetzbar. 

Zudem beschränkt sich das CSV-Format nur auf die Vermittlung der Daten und beinhaltet keine Informationen zur Darstellung der Tabelle. Excel-Dateien (XLS/XLSX) hingegen speichern auch Daten zur Formatierung, auch Zeilengrösse, Textgrösse, Textformat und viele mehr. Für die Zwecke dieser Applikation sind diese Informationen irrelevant.
 
 
Wenn die Applikation die CSV-Datei laden und verarbeiten will, dann braucht sie zunächst zusätzliche Informationen zu der Datei: Die \textit{URL}, ein Zeichenstring, der eine über das Internet zugängliche Ressource repräsentiert \cite{url}.

Die Applikation lädt die Datei durch eine \textit{Ajax}-Anfrage (\textit{Asynchronous JavaScript and XML}) herunter, bevor sie sie weiter verarbeitet.

\subsection{Filtering}

Das Filtering ist der Prozess, der Rohdaten in aufbereitete Daten umwandelt. Die Aufgaben des Filtering sind zum Beispiel die Vervollständigung, Reduzierung oder Korrektur der Daten, sodass sie in den folgenden Schritten der Visualisierungspipeline verwendet werden können \cite[Kapitel 2.1]{viz}.

\subsubsection*{Konvertierung in JavaScript-Objekte}

Als erster Schritt wird in der Applikation die CSV-Datei in \textit{JavaScript-Objekte} umgewandelt. Das Laden und \textit{Parsing} der Datei von CSV zu Objekten ist von D3 implementiert.

JavaScript-Objekte werden im \textit{JavaScript Object Notation}-Format (\textit{JSON}) dargestellt.

Im Abbildung \ref{fig:csv-json} ist der Prozess der Umwandlung ersichtlich: Es wird ein \textit{Array} von allen Zeilen in der Tabelle (ausgenommen der ersten Zeile, wo die Spalten beschriftet werden) erstellt. Jede Zeile wird als Objekt mit den dazugehörigen Spalten dargestellt.

\begin{figure}[!htbp]
	\centering
	\begin{minipage}{0.35\textwidth}
		\centering
		\begin{minted}[linenos]{text}
Date,Value
2010-12-31,4220.717
2009-12-31,4352.729
2008-12-31,5555.505
2007-12-31,5067.794
		\end{minted}
	\end{minipage}\hfill
	\begin{minipage}{0.55\textwidth}
		\centering

		\begin{minted}[linenos]{json}
[
  {
    "Date": "2010-12-31",
    "Value": "4220.717"
  },
  {
    "Date": "2009-12-31",
    "Value": "4352.729"
  },
  {
    "Date": "2008-12-31",
    "Value": "5555.505"
  },
  {
    "Date": "2007-12-31",
    "Value": "5067.794"
  }
]
		\end{minted}
	\end{minipage}
	\caption[CSV und JSON]{Konvertierung von CSV zu JavaScript Objekten. Links: CSV. Rechts: JSON.}
	\label{fig:csv-json}
\end{figure}

\subsubsection*{Formatierung}

Die Applikation benötigt nun Anweisungen, um den Datensatz (im JSON-Format) zu formatieren, damit er im Diagramm verwendet werden kann. Der Prozess muss folgende Aufgaben erledigen (diese Strategie wurde selber entwickelt):

\begin{itemize}
	\item (Zeichenstring-) Elemente gegebenfalls in JavaScript-Objekte umwandeln
	\item Falls mehrere Datensätze vorhanden sind, diese \textit{mergen}, also in einen einzigen Array zusammenfassen
	\item Den gesamten gemergten Datensatz nach der unabhängigen Variable aufsteigend sortieren
\end{itemize}

\paragraph{Umwandlung zu JavaScript-Objekten.} Das CSV-Format unterscheidet nicht zwischen Datentypen. Alle Werte in CSV-Dateien sind Zeichenstrings.

Im Beispiel in der Abbildung \ref{fig:csv-json} rechts, Zeile 3, wird für das erste Objekt im Array das Attribut mit dem Namen "`Date"' definiert. Der Datentyp ist hier ein Zeichenstring; damit ist eine Umwandlung in das JavaScript Date-Objekt, das ein Datum darstellt, sinnvoll: Das JavaScript Date-Objekt beherrscht viele Funktionen, wie zum Beispiel die Ausgabe der Anzahl Millisekunden, die seit dem 1. Januar 1970 vergangen sind. Dies ist beim Vergleichen von verschiedenen Date-Objekten nützlich. Das Date-Objekt ist zum Beispiel auch fähig, das Datum in einem Format auszugeben, das den lokalen Gegebenheiten entspricht: 28.10.2015 (Schweiz), 10/28/2015 (USA).

Zahlen, wie in Abbildung \ref{fig:csv-json} rechts, Zeile 4, im Attribut mit dem Namen "`Value"' definiert, werden zunächst von JavaScript als Zeichenstring behandelt. Diese Strings müssen in Zahlen-Objekte umgewandelt werden, denn nur mit Zahlen-Objekten können Rechenoperationen durchgeführt werden.

\paragraph{Merging von Datensätzen.} Oft werden mehrere Datensätze in der Applikation geladen. Ein Beispiel ist der $CO_2$-Verbrauch von Ländern: Für jedes Land wird eine separate CSV-Datei geladen und in einen JavaScript-Array umgewandelt.

Es wurde entschieden, alle geladenen Datensätze (Arrays) in ein Datensatz (Array) zusammenzufassen (\textit{merge}), weil die Umsetzung aus Sicht der Programmierlogik einfacher ist. 

Da Spalten von verschiedenen Datensätzen meist mit gleichem Namen beschriftet sind, könnte man nach dem Merge die Spalten nicht unterscheiden (Abbildung \ref{fig:merge} oben). Darum wird die Beschriftung aller Spalten der abhängigen Variablen durch eine eindeutige Identifikation ersetzt (Abbildung \ref{fig:merge} unten). Die eindeutige Identifikation wird durch den Spaltennamen und die URL des Datensatzes generiert: Dem Spaltennamen wird ein Rautenzeichen und die URL angehängt. Dies ermöglicht, dass man trotz Merge die Objekte dem ursprünglichen Datensatz zuordnen kann.

In der Abbildung \ref{fig:merge} wurden zwei Datensätze mit den Dateinamen ch-co2.csv und af-co2.csv gemergt. Im unteren Beispiel wurden die Spaltennamen der abhängigen Variablen durch die eindeutige Identifikation ersetzt.

\begin{figure}[!htbp]
	\centering
	\begin{minipage}{0.40\textwidth}
		\centering
		\begin{minted}[linenos]{json}
[
  {
    "Date": "2010-12-31",
    "Value": "4220.717"
  },
  {
    "Date": "2009-12-31",
    "Value": "4352.729"
  },
  {
    "Date": "2010-12-31",
    "Value": "1320.717"
  },
  {
    "Date": "2009-12-31",
    "Value": "7353.129"
  }
]
		\end{minted}
	\end{minipage}\hfill
	\begin{minipage}{0.5\textwidth}
		\centering
		\begin{minted}[linenos]{json}
[
  {
    "Date": "2010-12-31",
    "Value#ch-co2.csv": "4220.717"
  },
  {
    "Date": "2009-12-31",
    "Value#ch-co2.csv": "4352.729"
  },
  {
    "Date": "2010-12-31",
    "Value#af-co2.csv": "1320.717"
  },
  {
    "Date": "2009-12-31",
    "Value#af-co2.csv": "7353.129"
  }
]
		\end{minted}
	\end{minipage}
	\caption[Merge-Strategie]{Demonstration der Merge-Strategie und Anwendung der ID-Generierung. Oben: Gemergter Datensatz, ohne eindeutige IDs. Unten: Gemergter Datensatz, mit eindeutigen IDs.}
	\label{fig:merge}
\end{figure}

\paragraph{Sortieren des gemergten Datensatzes.} Der Datensatz wird nach der unabhängigen Variable ansteigend sortiert, damit die Berechnung von Interpolationen ermöglicht wird.

\subsubsection*{Datensatzspezifische Filteringkonfiguration und Hard Coding}

Die Applikation soll fähig sein, mehrere Formate von CSV-Datensätzen verarbeiten zu können. Der Benutzer sollte die Anweisungen zum Filtering anpassen können.

Die Implementation dieser Anweisungen zum Filtering (Spezifikation der URLs, abhängige und unabhängige Variablen, Datentypen der Spalten des Datensatzes) kann auf zwei Arten durchgeführt werden:

\begin{itemize}
	\item Die Anweisungen zum Filtering sind im Sinne des \textit{Hard Coding} implementiert
	\item Die Anweisungen zum Filtering sind in einer \textit{Konfiguration} abgelegt (\textit{Soft Coding})
\end{itemize}

\paragraph{Hard Coding.} Die Anweisungen (Konfiguration) zum Filtering sind direkt im Programmcode abgelegt. Falls Datensätze mit anderen URLs, Spaltennamen oder Datentypen in der Applikation verwendet werden sollen, so müssen die Anweisungen direkt im Programmcode entsprechend angepasst werden.

\paragraph{Soft Coding.} Beim Soft Coding sind die Anweisungen (Konfiguration) für das Filtering in einer externen Ressource abgelegt, zum Beispiel in einer Datei oder Datenbank. Die Applikation liest die Konfiguration und führt das Filtering dementsprechend aus. Dem Benutzer ist es so möglich, die Funktionsweise der Applikation anzupassen. Soft Coding wird in dieser Applikation verwendet.

\subsubsection*{meta.json}

Die Applikation verwendet als Konfiguration eine Datei im JSON-Format, die vor dem Filtering in ein JavaScript-Objekt umgewandelt werden kann. In Abbildung \ref{fig:meta} ist ein Beispiel der Konfigurationsdatei meta.json aufgelistet:

\begin{itemize}
	\item \texttt{dataset} (Zeile 3-22) beschreibt ein Datensatz
	\item \texttt{url} (Zeile 4) gibt die URL des Datensatzes an
	\item \texttt{config} (Zeile 5-21) ist ein Array, in dem die Spalten des Datensatzes konfiguriert werden
	\item \texttt{row} (Zeile 7, 14) gibt die zu konfigurierende Spalte an
	\item \texttt{type} (Zeile 8, 15) hat entweder den Wert \texttt{index} oder \texttt{value}. \texttt{index} bedeutet, dass die Spalte eine unabhängige, \texttt{value} bedeutet, dass die Spalte eine abhängige Variable ist
	\item \texttt{data\_type} (Zeile 9, 16) gibt den Datentyp der Spalte an, damit sie in das entsprechende JavaScript-Objekt umgewandelt werden kann. Er kann den Wert \texttt{Date} oder \texttt{Number} haben
	\item \texttt{date\_format} (Zeile 10) gibt bei Spalten mit Datentyp \texttt{Date} das Format des Datums an, damit es geparst werden kann
	\item \texttt{name} (Zeile 11, 17) setzt einen Namen für den Datensatz, der im Diagramm angezeigt werden soll
	\item \texttt{activated} (Zeile 18) gibt an, ob der Datensatz standardmässig im Diagramm angezeigt werden soll. Dieses Attribut spielt besonders eine Rolle bei Diagrammen, die mehrere Datensätze darstellen
	\item \texttt{unit} (Zeile 19) stellt die Einheit einer abhängigen Variablen dar. Sie wird als Information im Diagramm angezeigt
\end{itemize}

\begin{figure}[!htbp]
	\centering
	\begin{minted}[linenos]{json}
{
  "datasets": [
    {
      "url":"data/WWDI-AFG_EN_ATM_CO2E_KT.csv",
      "config": [
        {
          "row": "Date",
          "type": "index",
          "data_type": "Date",
          "date_format": "%Y-%m-%d",
          "name": "Datum"
        },
        {
          "row": "Value",
          "type": "value",
          "data_type": "Number",
          "name": "Afghanistan",
          "activated": true,
          "unit": "kt"
        }
      ]
    }
  ]	
}
	\end{minted}
	
	
	\caption[Beispiel der Konfigurationsdatei: meta.json]{\textbf{meta.json}: Beispiel der Konfigurationsdatei, die die Applikation verwendet.}
	\label{fig:meta}
\end{figure}

\subsection{Mapping}

Das Mapping ist der Prozess der Umwandlung von aufbereiteten Daten zu Geometriedaten, es ist das Kernstück der Diagrammerstellung. Es wird eine \textit{Daten-zu-Geometrie-Abbildung} realisiert \cite[Kapitel 2.1]{viz}.

In der Applikation, die in dieser Arbeit entwickelt wird, stellen die Geometriedaten die SVG-Elemente wie zum Beispiel Rechtecke, Kreise, Linien, Text dar.

Die Programmlogik des Mappings unterscheidet sich grundlegend bei jedem Diagrammtyp. In dieser Arbeit werden mehrere Typen von Diagrammen entwickelt, der Prozess des Mappings wird daher für jeden Typ separat behandelt.

\subsection{Rendering}

% geometriedaten -> bilddaten

Das Rendering ist der letzte Schritt der Diagrammerstellung, bei dem die Abbildung der Geometriedaten in Bilddaten erfolgt \cite[Kapitel 2.1]{viz}.

In unserem Falle sind Bilddaten das Diagramm, das im Browser angezeigt wird. Es ist die Aufgabe des Browsers, die Geometriedaten (SVG-Elemente) mit Darstellungsanweisungen (CSS) in Bilddaten (Pixel, angezeigte Grafik) zu rendern.
