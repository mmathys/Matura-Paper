\begin{abstract}
%In dieser Arbeit wurde untersucht, wie Punkt- und Liniendiagramme durch Interaktion verbessert werden können, sodass sich der Benutzer effizienter mit dem dargestellten Datensatz auseinandersetzen kann und die Analyse und das Verständnis der Daten erleichtert werden. 

%Eine Webapplikation für die Darstellung von 2- und 3-dimensionalen Datensätzen wurde entwickelt. Die Technologie einer Webapplikation wurde vorgestellt (JavaScript, HTML, CSS, SVG), das Applikationsdesign (Benutzeroberfläche, Konfiguration, Filtering, Mapping, Rendering) und Probleme, die während der Entwicklung auftraten, wurden dokumentiert. In dieser Arbeit wurden Konzepte der Datenvisualisierung (Visualisierungspipeline) und des Designs von Benutzeroberflächen (Information-Seeking Mantra) erläutert.

%Bei den entwickelten interaktiven Diagrammen wurden verschiedene Interaktionsmethoden implementiert, die geläufig sind (wie z.B. Zoom, Filter, Details auf Abruf) oder in der Praxis nur selten verwendet werden (Reduktion von drei- auf zweidimensionale Punktdiagramme durch Projektion).

In this paper it was investigated how point and line graphs can be improved by interaction, so that the user can deal more efficiently with the presented data and the analysis and the understanding of the data is simplified.

A web application for the display of 2- and 3-dimensional data sets was developed. The technology of a web application has been presented (JavaScript, HTML, CSS, SVG), the application design (UI, configuration, filtering, mapping, rendering) and problems which occurred during the development have been documented.

In this work the concepts of data visualization (Visualization Pipeline) and the design of user interfaces (Information-Seeking Mantra) were explained.

The developed interactive chart implements interaction methods which are common (such as Zoom, Filter, Details-on-demand) or rarely used in practice (reduction of three- to two-dimensional scatter plots by projection).\\

The full version of the paper (german) can be accessed at

\href{https://mmathys.github.io/maturapaper.pdf}{https://mmathys.github.io/maturapaper.pdf}
\end{abstract}