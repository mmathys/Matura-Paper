\chapter{Einleitung}
Die graphische Darstellung von Daten, das Diagramm, ist von grosser Bedeutung für die Gesellschaft. Man findet Diagramme überall: In etlichen Wissenschaften, wo sie nicht wegzudenken ist, in jeglichen Industrien, in Zeitungen, in Werbungen. In allen Fällen hat das Diagramm das gleiche Ziel: Die Vermittlung von Zusammenhängen und Informationen des Datensatzes.

Im Informationszeitalter sind Daten von immer grösserer Bedeutung, der Datenfluss vergrössert sich exponentiell mit der Zeit. Um Daten darstellen zu können, müssen sie zuerst gesammelt, sortiert und formatiert werden, bevor man mit der Auswertung beginnen kann. Das Sammeln von Daten stellt oftmals keine besondere Schwierigkeit dar, das Auswerten und Darstellen ist eine Herausforderung.

Diese Arbeit untersucht, wie der Informationsertrag eines Diagramm durch Interaktion erhöht werden kann. Im Web-Browser dargestellte Diagramme haben das Potential, interaktiv zu agieren. Man benutzt dafür Technologien wie HTML, CSS und JavaScript.

Für diese Arbeit wurde der Diagramm-Typ Scatterplot gewählt, weil dieser der am Meisten \cite{maxAmAbend} benutzte Diagramm-Typ ist.