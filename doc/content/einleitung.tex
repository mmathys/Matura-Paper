\chapter{Einleitung}
Die graphische Darstellung von Daten, das Diagramm, ist von grosser Bedeutung für die Gesellschaft. Man findet Diagramme überall: In etlichen Wissenschaften, wo sie nicht wegzudenken ist, in jeglichen Industrien, in Zeitungen, in Werbungen. In allen Fällen hat das Diagramm das gleiche Ziel: Die Vermittlung von Zusammenhängen und Informationen des Datensatzes.

Im Informationszeitalter sind Daten von immer grösserer Bedeutung, der Datenfluss vergrössert sich exponentiell mit der Zeit. Um Daten darstellen zu können, müssen sie zuerst gesammelt, sortiert und formatiert werden, bevor man mit der Auswertung beginnen kann. Das Sammeln von Daten stellt oftmals keine besondere Schwierigkeit dar, das Auswerten und Darstellen ist eine Herausforderung.

In dieser Arbeit werden Methoden entwickelt, um Informationen interaktiv in Diagrammen darzustellen. "`Konventionelle"' Diagramme sind statisch und können nicht vom Betrachter manipuliert werden, sie sind unveränderbar. Die interaktive Diagramme werden für den Web-Browser entwickelt, Technologien wie HTML (für den Inhalt), CSS (für das Aussehen), SVG (für Vektorgrafiken), JavaScript (für die Berechnung) werden verwendet, diese erlauben die dynamische Manipulation durch den Nutzer.

Mit der Entwicklung von Interaktionen soll das Diagramm zugänglicher gemacht werden und mehr Informationen übermittelt werden. Die Erkundung von Information ein vergnügliches Erlebnis sein und den Benutzer motivieren, sich mit den Daten auseinanderzusetzen.