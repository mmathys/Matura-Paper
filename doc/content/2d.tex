\section{Zweidimensionales Punktediagramm}

Da das zweidimensionale Punktediagramm weit verbreitet ist, sind sich Betrachter an diese Darstellung gewöhnt. Oft werden die Punkte in Diagramm durch eine Linie verbunden, was den Verlauf der abgebildeten Datenwerte verdeutlicht, besonders in Medien, Börsen. 

Als erstes Beispiel hat sich darum das zweidimensionale Punktediagramm, beziehungsweise falls Linien hinzugefügt werden das Liniendiagramm besonders geeignet.

\subsection{Information-Seeking Mantra}

Als Startreferenz für die Entwicklung des interaktiven Diagrammes bietet sich die von Ben Shneiderman begründeten Prinzipien für das Design graphischen Benutzeroberflächen an, die "`\textit{Information-Seeking Mantra}"'. Die Weise, wie der Benutzer mit der Oberfläche interagiert, hat Shneiderman \cite{shneiderman} festgelegt:

\begin{itemize}
	\item Überblick ("`\textit{Overview}"')
	\item Zoomen und Filtern ("`\textit{Zoom and Filter}"')
	\item Details auf Abruf ("`\textit{Details-on-demand}"')
\end{itemize}

% sagen: noch mehrere tasks, zb history. 