\documentclass[12pt]{document}  % default square logo 
%\documentclass[12pt,beltcrest]{ociamthesis} % use old belt crest logo
%\documentclass[12pt,shieldcrest]{ociamthesis} % use older shield crest logo

%load any additional packages
\usepackage{amssymb}
% Korrektes Kodieren des in und outputs
\usepackage[T1]{fontenc}
\usepackage[utf8]{inputenc}
\usepackage{url}

% for general code highlighting
\usepackage{minted}

% for inline code highlighting
\usepackage{pythontex}

\usepackage[backend=biber,style=numeric]{biblatex}
\addbibresource{refs.bib}

%Deutsche Silbentrennung
\RequirePackage{hyphsubst}%
\HyphSubstIfExists{ngerman-x-latest}{\HyphSubstLet{ngerman}{ngerman-x-latest}}{}

%shorthand für tiefstellen
\usepackage{subscript}

\usepackage{mathtools}


%input macros (i.e. write your own macros file called mymacros.tex 
%and uncomment the next line)
%\include{mymacros}

\title{Entwicklung von\\[1ex]     %your thesis title,
       Diagrammen im Web}   %note \\[1ex] is a line break in the title

\author{Max Mathys}             %your name
\college{MNG Rämibühl}  %your college

\renewcommand{\submittedtext}{4. Januar 2016}
\renewcommand{\supervisor}{Betreuende Lehrperson: David Sichau}
\degree{Maturitätsarbeit}     %the degree
\degreedate{2016}         %the degree date


\usepackage{biblatex}

% \bibliography{<mybibfile>}% ONLY selects .bib file; syntax for version <= 1.1b
\addbibresource{refs.bib}% Syntax for version >= 1.2



%end the preamble and start the document
\begin{document}

%this baselineskip gives sufficient line spacing for an examiner to easily
%markup the thesis with comments
\baselineskip=18pt plus1pt

%set the number of sectioning levels that get number and appear in the contents
\setcounter{secnumdepth}{3}
\setcounter{tocdepth}{3}


\maketitle                  % create a title page from the preamble info
\begin{abstract}
In dieser Arbeit wurde untersucht, wie Punkt- und Liniendiagramme durch Interaktion verbessert werden können, sodass sich der Benutzer effizienter mit dem dargestellten Datensatz auseinandersetzen kann und die Analyse und das Verständnis der Daten erleichtert werden. 

Eine Webapplikation für die Darstellung von 2- und 3-dimensionalen Datensätzen wurde entwickelt. Die Technologie einer Webapplikation wurde vorgestellt (JavaScript, HTML, CSS, SVG), das Applikationsdesign (Benutzeroberfläche, Konfiguration, Filtering, Mapping, Rendering) und Probleme, die während der Entwicklung auftraten, wurden dokumentiert. In dieser Arbeit wurden Konzepte der Datenvisualisierung (Visualisierungspipeline) und des Designs von Benutzeroberflächen (Information-Seeking Mantra) erläutert.

Bei den entwickelten interaktiven Diagrammen wurden verschiedene Interaktionsmethoden implementiert, die geläufig sind (wie z.B. Zoom, Filter, Details auf Abruf) oder in der Praxis nur selten verwendet werden (Reduktion von drei- auf zweidimensionale Punktdiagramme durch Projektion).
\end{abstract}       % include the abstract


\begin{romanpages}          % start roman page numbering
\tableofcontents            % generate and include a table of contents
\end{romanpages}            % end roman page numbering

%now include the files of latex for each of the chapters etc
\chapter{Einleitung}
Der Anwendungsbereich von Diagrammen ist sehr vielseitig. Man kann in vielen verschiedenen Industrien 
\chapter{Hauptteil}

Nachstehend wird beschrieben, wie die Applikation entwickelt wurde.

\section{Technologie}

Für die Entwicklung einer Applikation für den Web-Browser werden verschiedene Technologien benötigt, die jeweils für einen einzelnen Aspekt der Applikation verantwortlich sind.

\subsection{JavaScript}

JavaScript ist die Programmiersprache, die der Web-Browser unterstützt. Die Programmlogik wird in dieser Sprache geschrieben.

Für bessere Programmqualität und Lesbarkeit wird in dieser Arbeit der \textit{JavaScript Standard Style} gebraucht, eine verbreitete Syntaxkonvention. Viele Programmierer tendieren dazu, den JavaScript Code nach persönlicher Weise und auf nicht konsequente Weise zu formatieren, was die Lesbarkeit des Programmcodes, für den Autor als auch besonders für andere Betrachter, verschlechtert. Dies kann sich vor allem in umfangreichen Projekten, wo mehrere hundert Zeilen Code vorhanden sind, markant auf die Übersichtlichkeit auswirken. Da Open-Source von mehr und mehr praktiziert wird, gewinnen Konvetionen an Bedeutung. "`Das Anwenden der [Syntax-] Konvention bedeutet, die Syntaxkonventionen der Community und den Belang der Lesbarkeit des Programmcodes über die persönliche Programmierweise zu stellen"' \cite{feross}.

Das Produkt dieser Arbeit verwendet in allen Tests die Programmierbibliothek \textit{Data-Driven-Documents (D3)}. Sie wurde von Michael Bostock, Vadim Ogievetsky und Jeffrey Heer erstellt und dient zur Entwicklung von Visualisationen im Web. Sie erleichtert die Benutzung des Document Object Models (DOM) (vgl. Abbildung <todo>), ermöglicht effizienteres Debugging und verbessert die Leistung (\textit{"`Performance"'}) der Applikation \cite{bostock}.

\subsection{HTML, Document Object Model (DOM)}

HTML (Hypertext Markup Language) ist in der Entwicklung einer Web-Applikation für den Inhalt der Seite zuständig. Die Sprache beschreibt durch \textit{Elemente}, die einen Wert haben können, verschachtelt sein können und denen \textit{Attribute} zugewiesen werden können die zu darstellenden Informationen. Das Document Object Model (DOM) ermöglicht die dynamische Manipulation dieser Elemente durch Schnittstellen in JavaScript-Programmen.

\subsection{Scalable Vector Graphics (SVG)}

SVG (Scalable Vector Graphics) ist ein Format für Vektorgrafiken. SVG-Bilder bestehen nicht wie andere Bilderformate (JPG, PNG) aus Pixeln, sondern aus Elementen wie Kreise, Ellipsen, Rechtecke, Linien. SVG-Grafiken können im Browser dargestellt werden und durch JavaScript ebenfalls dynamisch manipuliert werden.

\subsection{Cascading Style Sheets (CSS)}

Cascading Style Sheets (CSS) beschreiben die Darstellung der anzuzeigenden Elemente, die in HTML-Dokumenten oder SVG-Grafiken vorkommen. CSS-Attribute können in HTML- oder SVG-Elementen im Attribut \lstinline[language=html]{class} durch \textit{Klassen} zugewiesen werden oder direkt im Attribut \lstinline[language=html]{style} definiert werden.
\section{Verarbeitung der Daten}

Bevor das Diagramm im Browser dargestellt werden kann, müssen zuerst die Daten geladen verarbeitet werden. Die Beispiele dieser Arbeit nutzen grösstenteils Datensätze, die frei verfügbar sind.  %hier näher erklären TODO
Die Diagramme in Abbildung \ref{fig:scatterplot} benutzen einen Datensatz der Weltbank, der den $CO_2$-Verbrauch von Afghanistan beinhaltet.

Beim Prozess zur Veranschaulichung von Daten wird die \textit{Visualisierungspipeline} durchlaufen \cite{pipeline}. Diese Pipeline stellt drei wesentliche Schritte des Prozesses dar: Die Datenaufbereitung (\textit{Filtering}), die Erzeugung des Geometriemodells (\textit{Mapping}) und die Bildgenerierung (\textit{Rendering}).

% = rohdaten -> aufbereitete Daten -> geometriedaten -> Bilddaten

\subsection{Rohdaten}

\subsubsection{Comma-seperated values (CSV)}

Die Software verwendet als Datenformat \textit{Comma-seperated values} (\textit{CSV}). CSV stellt eine Tabelle dar. Zeilen werden im Format durch einen Umbruch und Spaltenwerte durch ein Komma getrennt. Optional steht in der ersten Zeile (Abbildung \ref{fig:csv} links, Zeile 1) der Datei die Beschriftung der Spalten.

In Abbildung \ref{fig:csv} wird die Funktion des CSV-Formats demonstriert.

\begin{figure}[!htbp]
	\centering
	\begin{minipage}{0.5\textwidth}
		\centering
		\begin{minted}{text}
Date,Value
2010-12-31,4220.717
2009-12-31,4352.729
2008-12-31,5555.505
2007-12-31,5067.794
		\end{minted}
	\end{minipage}\hfill
	\begin{minipage}{0.5\textwidth}
		\centering
		\begin{tabular}{ | l | l |}
			\hline
			\textbf{Date} & \textbf{Value} \\ \hline
			2010-21-31 & 4220.717 \\ \hline
			2009-21-31 & 4352.729 \\ \hline
			2008-21-31 & 5555.505 \\ \hline
			2007-21-31 & 5067.794 \\ \hline
		\end{tabular}
	\end{minipage}
	\caption[Demonstration des CSV-Formats]{Demonstration des CSV-Formats. Links: Die CSV-Datei in Rohtext. Rechts: Darstellung der Informationen der CSV-Datei in einer Tabelle.}
	\label{fig:csv}
\end{figure}

Man verwendet das CSV-Format oft, weil es sehr einfach aufgebaut ist. Das Lesen von solchen Tabellenformaten ist ohne grossen Aufwand in Programmen umsetzbar. 

 Zudem beschränkt sich das CSV-Format nur auf die Vermittlung der Daten, und beinhaltet keine Informationen zu der Darstellung der Tabelle. Excel-Dateien (XLS/XLSX) hingegen speichern auch Daten zu der Formatierung, auch Zeilengrösse, Textgrösse, Textformat und viele mehr. Für die Zwecke dieser Software sind diese Informationen irrelevant.
 
 
Wenn die Software die CSV-Datei laden und verarbeiten will, dann braucht sie zunächst zusätzliche Informationen zu der Datei: Die \textit{URL}, die den Ort der Datei angibt.

Die Software lädt die Datei durch eine \textit{Ajax}-Anfrage (\textit{Asynchronous JavaScript and XML}) herunter, bevor sie sie weiter verarbeitet.

\subsection{Filtering}

Das Filtering ist der Prozess, der Rohdaten in aufbereitete Daten umwandelt. Die Aufgaben des Filterings sind zum Beispiel die Vervollständigung, Reduzierung, Korrektur der Daten, sodass sie in den folgenden Schritten der Visualisationspipeline verwendet werden können \cite[Kap. 2]{viz}.

\subsubsection{Konvertierung in JavaScript Objects}

Als erster Schritt wird in der Applikation die CSV-Datei in \textit{JavaScript-Objekte} umgewandelt. Das Laden und \textit{Parsing} der Datei von CSV zu Objekten ist von D3 implementiert.

JavaScript Objekte werden im \textit{JavaScript Object Notation}-Format (\textit{JSON}) dargestellt.

Im Abbildung \ref{fig:csv-json} ist der Prozess der Umwandlung ersichtlich: Es wird ein \textit{Array} von allen Zeilen in der Tabelle (ausgenommen der ersten Zeile, wo die Spalten beschriftet werden) erstellt. Jede Zeile wird als Objekt mit den dazugehörigen Spalten dargestellt.

\begin{figure}[!htbp]
	\centering
	\begin{minipage}{0.35\textwidth}
		\centering
		\begin{minted}{text}
Date,Value
2010-12-31,4220.717
2009-12-31,4352.729
2008-12-31,5555.505
2007-12-31,5067.794
		\end{minted}
	\end{minipage}\hfill
	\begin{minipage}{0.55\textwidth}
		\centering

		\begin{minted}{json}
[
  {
    "Date": "2010-12-31",
    "Value": "4220.717"
  },
  {
    "Date": "2009-12-31",
    "Value": "4352.729"
  },
  {
    "Date": "2008-12-31",
    "Value": "5555.505"
  },
  {
    "Date": "2007-12-31",
    "Value": "5067.794"
  }
]
		\end{minted}
	\end{minipage}
	\caption[CSV und JSON]{Konvertierung von CSV zu JavaScript Objekten. Links: CSV. Rechts: JSON.}
	\label{fig:csv-json}
\end{figure}

\subsubsection{Formatierung}

Die Software benötigt nun Anweisungen, um den Datensatz (im JSON-Format) zu formatieren, damit er im Diagramm verwendet werden kann. Der Prozess muss folgende Aufgaben erledigen:

\begin{itemize}
	\item (Zeichenstring-) Elemente gegebenfalls in JavaScript-Objekte umwandeln
	\item Falls mehrere Datensätze vorhanden sind, diese \textit{mergen}, also in einen einzigen Array zusammenfassen
	\item Den gesamten gemergten Datensatz nach der unabhängigen Variable aufsteigend sortieren
\end{itemize}

\textbf{Umwandlung zu JavaScript-Objekten.} Das CSV-Format unterscheidet nicht zwischen Datentypen. Alle Werte in CSV-Dateien sind Zeichenstrings.

In dem Beispiel in der Abbildung \ref{fig:csv-json} rechts, Zeile 3 wird für das erste Objekt im Array das Attribut mit dem Namen "`Date"' definiert. Der Datentyp ist hier ein Zeichenstring, eine Umwandlung in das JavaScript-Date-Objekt, das ein Datum darstellt, ist sinnvoll: Das JavaScript Date-Objekt beherrscht viele Funktionen, wie zum Beispiel die Ausgabe der Anzahl Millisekunden, die seit dem 1. Januar 1970 vergangen sind. Dies ist beim Vergleichen von verschiedenen Date-Objekten nützlich. Das Date-Objekt ist zum Beispiel auch fähig, das Datum in einem Format auszugeben, das den lokalen Konventionen entspricht: 28.10.2015 (Schweiz), 10/28/2015 (USA).

Nummern, wie in Abbildung \ref{fig:csv-json} rechts, Zeile 4, im Attribut mit dem Namen "`Value"' definiert, wandeln wir in Nummern um. Nur an Nummer-Objekten können Rechenoperationen durchgeführt werden.

\textbf{Merging von Datensätzen.} Oft werden mehrere Datensätze in der Software geladen. Ein Beispiel ist der $CO_2$-Verbrauch von Ländern: Für jedes Land wird eine separate CSV-Datei geladen und in ein JavaScript-Array umgewandelt.

Ich entschieden, alle geladenen Datensätze (Arrays) in ein Datensatz (Array) zusammenzufassen (\textit{merge}), weil dies für die Programmlogik mehr Sinn macht.

Da Spalten von verschiedenen Datensätzen meist mit gleichem Namen beschriftet sind, könnte man nach dem Merge die Spalten nicht unterscheiden (Abbildung \ref{fig:merge} oben). Darum wird die Beschriftung aller Spalten der abhängigen Variablen durch eine eindeutige ID ersetzt (Abbildung \ref{fig:merge} unten). Die eindeutige ID wird durch den Spaltennamen und die URL der Datensatzes generiert: Der Spaltennamen wird ein Rautenzeichen und die URL angehängt. Dies ermöglicht, dass man trotz Merge die Objekte dem herkömmlichen Datensatz zuordnen kann.

In der Abbildung \ref{fig:merge} wurden zwei Datensätze mit dem Dateinamen ch-co2.csv und af-co2.csv gemergt. Im unteren Beispiel wurden die Spaltennamen der abhängigen Variablen durch die eindeutige ID ersetzt.

\begin{figure}[!htbp]
	\centering
	\begin{minipage}{0.40\textwidth}
		\centering
		\begin{minted}{json}
[
  {
    "Date": "2010-12-31",
    "Value": "4220.717"
  },
  {
    "Date": "2009-12-31",
    "Value": "4352.729"
  },
  {
    "Date": "2010-12-31",
    "Value": "1320.717"
  },
  {
    "Date": "2009-12-31",
    "Value": "7353.129"
  }
]
		\end{minted}
	\end{minipage}\hfill
	\begin{minipage}{0.5\textwidth}
		\centering
		\begin{minted}{json}
[
  {
    "Date": "2010-12-31",
    "Value#ch-co2.csv": "4220.717"
  },
  {
    "Date": "2009-12-31",
    "Value#ch-co2.csv": "4352.729"
  },
  {
    "Date": "2010-12-31",
    "Value#af-co2.csv": "1320.717"
  },
  {
    "Date": "2009-12-31",
    "Value#af-co2.csv": "7353.129"
  }
]
		\end{minted}
	\end{minipage}
	\caption[Merge-Strategie]{Demonstration der Merge-Strategie und Anwendung der ID-Generierung. Oben: Gemergter Datensatz, ohne eindeutige IDs. Unten: Gemergter Datensatz, mit eindeutigen IDs.}
	\label{fig:merge}
\end{figure}

\textbf{Sortieren vom gemergten Datensatz.} Der Datensatz wird nach der unabhängigen Variable ansteigend sortiert, damit die Berechnung von Interpolationen ermöglicht werden.

\subsubsection{Datensatzspezifische Konfiguration und Hard Coding}

\subsection{Aufbereitete Daten}

\subsection{Mapping}

\subsection{Geometriedaten}

\subsection{Rendering}

\subsection{Bilddaten}
	\section{Technologie}

Für die Entwicklung einer Applikation für den Web-Browser werden verschiedene Technologien benötigt, die jeweils für einen einzelnen Aspekt der Applikation verantwortlich sind.

\subsection{JavaScript}

JavaScript ist die Programmiersprache, die der Web-Browser unterstützt. Die Programmlogik wird in dieser Sprache geschrieben.

Für bessere Programmqualität und Lesbarkeit wird in dieser Arbeit der \textit{JavaScript Standard Style} gebraucht, eine verbreitete Syntaxkonvention. Viele Programmierer tendieren dazu, den JavaScript Code nach persönlicher Weise und auf nicht konsequente Weise zu formatieren, was die Lesbarkeit des Programmcodes, für den Autor als auch besonders für andere Betrachter, verschlechtert. Dies kann sich vor allem in umfangreichen Projekten, wo mehrere hundert Zeilen Code vorhanden sind, markant auf die Übersichtlichkeit auswirken. Da Open-Source von mehr und mehr praktiziert wird, gewinnen Konvetionen an Bedeutung. "`Das Anwenden der [Syntax-] Konvention bedeutet, die Syntaxkonventionen der Community und den Belang der Lesbarkeit des Programmcodes über die persönliche Programmierweise zu stellen"' \cite{feross}.

Das Produkt dieser Arbeit verwendet in allen Tests die Programmierbibliothek \textit{Data-Driven-Documents (D3)}. Sie wurde von Michael Bostock, Vadim Ogievetsky und Jeffrey Heer erstellt und dient zur Entwicklung von Visualisationen im Web. Sie erleichtert die Benutzung des Document Object Models (DOM) (vgl. Abbildung <todo>), ermöglicht effizienteres Debugging und verbessert die Leistung (\textit{"`Performance"'}) der Applikation \cite{bostock}.

\subsection{HTML, Document Object Model (DOM)}

HTML (Hypertext Markup Language) ist in der Entwicklung einer Web-Applikation für den Inhalt der Seite zuständig. Die Sprache beschreibt durch \textit{Elemente}, die einen Wert haben können, verschachtelt sein können und denen \textit{Attribute} zugewiesen werden können die zu darstellenden Informationen. Das Document Object Model (DOM) ermöglicht die dynamische Manipulation dieser Elemente durch Schnittstellen in JavaScript-Programmen.

\subsection{Scalable Vector Graphics (SVG)}

SVG (Scalable Vector Graphics) ist ein Format für Vektorgrafiken. SVG-Bilder bestehen nicht wie andere Bilderformate (JPG, PNG) aus Pixeln, sondern aus Elementen wie Kreise, Ellipsen, Rechtecke, Linien. SVG-Grafiken können im Browser dargestellt werden und durch JavaScript ebenfalls dynamisch manipuliert werden.

\subsection{Cascading Style Sheets (CSS)}

Cascading Style Sheets (CSS) beschreiben die Darstellung der anzuzeigenden Elemente, die in HTML-Dokumenten oder SVG-Grafiken vorkommen. CSS-Attribute können in HTML- oder SVG-Elementen im Attribut \lstinline[language=html]{class} durch \textit{Klassen} zugewiesen werden oder direkt im Attribut \lstinline[language=html]{style} definiert werden.
	\section{Verarbeitung der Daten}

Bevor das Diagramm im Browser dargestellt werden kann, müssen zuerst die Daten geladen verarbeitet werden. Die Beispiele dieser Arbeit nutzen grösstenteils Datensätze, die frei verfügbar sind.  %hier näher erklären TODO
Die Diagramme in Abbildung \ref{fig:scatterplot} benutzen einen Datensatz der Weltbank, der den $CO_2$-Verbrauch von Afghanistan beinhaltet.

Beim Prozess zur Veranschaulichung von Daten wird die \textit{Visualisierungspipeline} durchlaufen \cite{pipeline}. Diese Pipeline stellt drei wesentliche Schritte des Prozesses dar: Die Datenaufbereitung (\textit{Filtering}), die Erzeugung des Geometriemodells (\textit{Mapping}) und die Bildgenerierung (\textit{Rendering}).

% = rohdaten -> aufbereitete Daten -> geometriedaten -> Bilddaten

\subsection{Rohdaten}

\subsubsection{Comma-seperated values (CSV)}

Die Software verwendet als Datenformat \textit{Comma-seperated values} (\textit{CSV}). CSV stellt eine Tabelle dar. Zeilen werden im Format durch einen Umbruch und Spaltenwerte durch ein Komma getrennt. Optional steht in der ersten Zeile (Abbildung \ref{fig:csv} links, Zeile 1) der Datei die Beschriftung der Spalten.

In Abbildung \ref{fig:csv} wird die Funktion des CSV-Formats demonstriert.

\begin{figure}[!htbp]
	\centering
	\begin{minipage}{0.5\textwidth}
		\centering
		\begin{minted}{text}
Date,Value
2010-12-31,4220.717
2009-12-31,4352.729
2008-12-31,5555.505
2007-12-31,5067.794
		\end{minted}
	\end{minipage}\hfill
	\begin{minipage}{0.5\textwidth}
		\centering
		\begin{tabular}{ | l | l |}
			\hline
			\textbf{Date} & \textbf{Value} \\ \hline
			2010-21-31 & 4220.717 \\ \hline
			2009-21-31 & 4352.729 \\ \hline
			2008-21-31 & 5555.505 \\ \hline
			2007-21-31 & 5067.794 \\ \hline
		\end{tabular}
	\end{minipage}
	\caption[Demonstration des CSV-Formats]{Demonstration des CSV-Formats. Links: Die CSV-Datei in Rohtext. Rechts: Darstellung der Informationen der CSV-Datei in einer Tabelle.}
	\label{fig:csv}
\end{figure}

Man verwendet das CSV-Format oft, weil es sehr einfach aufgebaut ist. Das Lesen von solchen Tabellenformaten ist ohne grossen Aufwand in Programmen umsetzbar. 

 Zudem beschränkt sich das CSV-Format nur auf die Vermittlung der Daten, und beinhaltet keine Informationen zu der Darstellung der Tabelle. Excel-Dateien (XLS/XLSX) hingegen speichern auch Daten zu der Formatierung, auch Zeilengrösse, Textgrösse, Textformat und viele mehr. Für die Zwecke dieser Software sind diese Informationen irrelevant.
 
 
Wenn die Software die CSV-Datei laden und verarbeiten will, dann braucht sie zunächst zusätzliche Informationen zu der Datei: Die \textit{URL}, die den Ort der Datei angibt.

Die Software lädt die Datei durch eine \textit{Ajax}-Anfrage (\textit{Asynchronous JavaScript and XML}) herunter, bevor sie sie weiter verarbeitet.

\subsection{Filtering}

Das Filtering ist der Prozess, der Rohdaten in aufbereitete Daten umwandelt. Die Aufgaben des Filterings sind zum Beispiel die Vervollständigung, Reduzierung, Korrektur der Daten, sodass sie in den folgenden Schritten der Visualisationspipeline verwendet werden können \cite[Kap. 2]{viz}.

\subsubsection{Konvertierung in JavaScript Objects}

Als erster Schritt wird in der Applikation die CSV-Datei in \textit{JavaScript-Objekte} umgewandelt. Das Laden und \textit{Parsing} der Datei von CSV zu Objekten ist von D3 implementiert.

JavaScript Objekte werden im \textit{JavaScript Object Notation}-Format (\textit{JSON}) dargestellt.

Im Abbildung \ref{fig:csv-json} ist der Prozess der Umwandlung ersichtlich: Es wird ein \textit{Array} von allen Zeilen in der Tabelle (ausgenommen der ersten Zeile, wo die Spalten beschriftet werden) erstellt. Jede Zeile wird als Objekt mit den dazugehörigen Spalten dargestellt.

\begin{figure}[!htbp]
	\centering
	\begin{minipage}{0.35\textwidth}
		\centering
		\begin{minted}{text}
Date,Value
2010-12-31,4220.717
2009-12-31,4352.729
2008-12-31,5555.505
2007-12-31,5067.794
		\end{minted}
	\end{minipage}\hfill
	\begin{minipage}{0.55\textwidth}
		\centering

		\begin{minted}{json}
[
  {
    "Date": "2010-12-31",
    "Value": "4220.717"
  },
  {
    "Date": "2009-12-31",
    "Value": "4352.729"
  },
  {
    "Date": "2008-12-31",
    "Value": "5555.505"
  },
  {
    "Date": "2007-12-31",
    "Value": "5067.794"
  }
]
		\end{minted}
	\end{minipage}
	\caption[CSV und JSON]{Konvertierung von CSV zu JavaScript Objekten. Links: CSV. Rechts: JSON.}
	\label{fig:csv-json}
\end{figure}

\subsubsection{Formatierung}

Die Software benötigt nun Anweisungen, um den Datensatz (im JSON-Format) zu formatieren, damit er im Diagramm verwendet werden kann. Der Prozess muss folgende Aufgaben erledigen:

\begin{itemize}
	\item (Zeichenstring-) Elemente gegebenfalls in JavaScript-Objekte umwandeln
	\item Falls mehrere Datensätze vorhanden sind, diese \textit{mergen}, also in einen einzigen Array zusammenfassen
	\item Den gesamten gemergten Datensatz nach der unabhängigen Variable aufsteigend sortieren
\end{itemize}

\textbf{Umwandlung zu JavaScript-Objekten.} Das CSV-Format unterscheidet nicht zwischen Datentypen. Alle Werte in CSV-Dateien sind Zeichenstrings.

In dem Beispiel in der Abbildung \ref{fig:csv-json} rechts, Zeile 3 wird für das erste Objekt im Array das Attribut mit dem Namen "`Date"' definiert. Der Datentyp ist hier ein Zeichenstring, eine Umwandlung in das JavaScript-Date-Objekt, das ein Datum darstellt, ist sinnvoll: Das JavaScript Date-Objekt beherrscht viele Funktionen, wie zum Beispiel die Ausgabe der Anzahl Millisekunden, die seit dem 1. Januar 1970 vergangen sind. Dies ist beim Vergleichen von verschiedenen Date-Objekten nützlich. Das Date-Objekt ist zum Beispiel auch fähig, das Datum in einem Format auszugeben, das den lokalen Konventionen entspricht: 28.10.2015 (Schweiz), 10/28/2015 (USA).

Nummern, wie in Abbildung \ref{fig:csv-json} rechts, Zeile 4, im Attribut mit dem Namen "`Value"' definiert, wandeln wir in Nummern um. Nur an Nummer-Objekten können Rechenoperationen durchgeführt werden.

\textbf{Merging von Datensätzen.} Oft werden mehrere Datensätze in der Software geladen. Ein Beispiel ist der $CO_2$-Verbrauch von Ländern: Für jedes Land wird eine separate CSV-Datei geladen und in ein JavaScript-Array umgewandelt.

Ich entschieden, alle geladenen Datensätze (Arrays) in ein Datensatz (Array) zusammenzufassen (\textit{merge}), weil dies für die Programmlogik mehr Sinn macht.

Da Spalten von verschiedenen Datensätzen meist mit gleichem Namen beschriftet sind, könnte man nach dem Merge die Spalten nicht unterscheiden (Abbildung \ref{fig:merge} oben). Darum wird die Beschriftung aller Spalten der abhängigen Variablen durch eine eindeutige ID ersetzt (Abbildung \ref{fig:merge} unten). Die eindeutige ID wird durch den Spaltennamen und die URL der Datensatzes generiert: Der Spaltennamen wird ein Rautenzeichen und die URL angehängt. Dies ermöglicht, dass man trotz Merge die Objekte dem herkömmlichen Datensatz zuordnen kann.

In der Abbildung \ref{fig:merge} wurden zwei Datensätze mit dem Dateinamen ch-co2.csv und af-co2.csv gemergt. Im unteren Beispiel wurden die Spaltennamen der abhängigen Variablen durch die eindeutige ID ersetzt.

\begin{figure}[!htbp]
	\centering
	\begin{minipage}{0.40\textwidth}
		\centering
		\begin{minted}{json}
[
  {
    "Date": "2010-12-31",
    "Value": "4220.717"
  },
  {
    "Date": "2009-12-31",
    "Value": "4352.729"
  },
  {
    "Date": "2010-12-31",
    "Value": "1320.717"
  },
  {
    "Date": "2009-12-31",
    "Value": "7353.129"
  }
]
		\end{minted}
	\end{minipage}\hfill
	\begin{minipage}{0.5\textwidth}
		\centering
		\begin{minted}{json}
[
  {
    "Date": "2010-12-31",
    "Value#ch-co2.csv": "4220.717"
  },
  {
    "Date": "2009-12-31",
    "Value#ch-co2.csv": "4352.729"
  },
  {
    "Date": "2010-12-31",
    "Value#af-co2.csv": "1320.717"
  },
  {
    "Date": "2009-12-31",
    "Value#af-co2.csv": "7353.129"
  }
]
		\end{minted}
	\end{minipage}
	\caption[Merge-Strategie]{Demonstration der Merge-Strategie und Anwendung der ID-Generierung. Oben: Gemergter Datensatz, ohne eindeutige IDs. Unten: Gemergter Datensatz, mit eindeutigen IDs.}
	\label{fig:merge}
\end{figure}

\textbf{Sortieren vom gemergten Datensatz.} Der Datensatz wird nach der unabhängigen Variable ansteigend sortiert, damit die Berechnung von Interpolationen ermöglicht werden.

\subsubsection{Datensatzspezifische Konfiguration und Hard Coding}

\subsection{Aufbereitete Daten}

\subsection{Mapping}

\subsection{Geometriedaten}

\subsection{Rendering}

\subsection{Bilddaten}
\chapter{Schlusswort}

Durch Umsetzung von verschiedenen Interaktionsmethoden an selbstentwickelten Diagrammen konnte die Analyse, Verständnis an zwei von drei Diagrammtypen gegenüber der statischen Version verbessert werden.

Neben der Dokumentation des Entwicklungsprozesses und der verwendeten Technologien wurde für die interaktiven Diagramme eine optimale Datenverarbeitungsstragie entwickelt (siehe "`Merge"' und "`meta.json"'). Nebenbei wurde auch viel Wissen über JavaScript im Allgemeinen, JavaScript Buildsystemen (NPM, Gulp.js, Browserify), Open Source, Web Design (CSS, Minimalismus, Pop/Unpop) und über den Umgang mit Programmbibliotheken (D3, three.js, tween.js) erarbeitet.

Beim zweidimensionalen Punkt-/Liniendiagamm wurden Möglichkeiten für die Interaktion mit Benutzeroberflächen entsprechend implementiert: Zoom, Tooltip, Detailanzeige, Datensatzauswahl. Es wurde so auch erreicht, dass Benutzer sich mit einer grossen Datenmenge effizienter auseinandersetzen können. Zudem wurden Diagrammtechniken implementiert wie Achsen, Skalierungen, Interpolationen. Die Theorie hinter diese Möglichkeiten und Techniken sind ebenfalls in der Arbeit dokumentiert worden.

Eine unkonventionelle Weise der Darstellung eines Datensatzes mit zwei abhängigen Variablen, das dreidimensionale Diagramm, wurde erläutert und als Applikation umgesetzt. Eine Methode wurde dazu entwickelt, die die Benutzung des Diagramms produktiver machen sollte: Orthographische Projektionen, die das dreidimensionale Punktdiagramm auf zweidimensionale Punktdiagramme reduzieren können.

Ein N-dimensionales Punktdiagramm wurde umgesetzt, doch es wurde erkannt, dass sich dieses Diagramm nicht als interaktives Diagramm eignet.




\newpage
%now enable appendix numbering format and include any appendices
\appendix

%next line adds the Bibliography to the contents page
%\renewcommand{\bibname}{Literatur}
\chapter{Literaturverzeichnis}
\printbibliography[heading=none]     %use a bibtex bibliography file refs.bib

\newpage
\chapter{Abbildungsverzeichnis}
\makeatletter
\@starttoc{lof} % Print List of Figures
\makeatother

\newpage
\input{content/bestatigung}

\end{document}